\section{Wupper register map}
\label{App:Regmap}
Starting from the offset address of BAR0, BAR1 and BAR2, the register map for BAR0 expands from 0x0000 to 0x0430 for the PCIe control registers. BAR0 only contains registers associated with DMA. The offset for BAR0 is usually 0xFBB00000.

\begin{longtabu} to \textwidth {|X[1.5,l]|X[0.8,l]|X[3,l]|X[4,l]|X[1,l]|X[1,l]|X[5,l]|}
\hline
\textbf{Address} &\textbf{PCIe} &\multicolumn{2}{|l|}{\textbf{Name/Field}} &\textbf{Bits} &{\textbf{Type}} &\textbf{Description} \\
\hline
\endhead

\multicolumn{7}{|c|}{Bar0} \\
\hline
\multicolumn{7}{|c|}{DMA\_DESC} \\
\hline
0x0000 & 0,1 & \multicolumn{5}{|l|}{DMA\_DESC\_0} \\
\cline{3-7}
 & & & END\_ADDRESS & 127:64 & W & End Address \\
 & & & START\_ADDRESS & 63:0 & W & Start Address \\
\hline
0x0010 & 0,1 & \multicolumn{5}{|l|}{DMA\_DESC\_0a} \\
\cline{3-7}
 & & & RD\_POINTER & 127:64 & W & PC Read Pointer \\
 & & & WRAP\_AROUND & 12 & W & Wrap around \\
 & & & READ\_WRITE & 11 & W & 1: fromHost/ 0: toHost \\
 & & & NUM\_WORDS & 10:0 & W & Number of 32 bit words \\
\hline
\multicolumn{7}{|c|}{\ldots} \\
\hline
0x00E0 & 0,1 & \multicolumn{5}{|l|}{DMA\_DESC\_7} \\
\cline{3-7}
 & & & END\_ADDRESS & 127:64 & W & End Address \\
 & & & START\_ADDRESS & 63:0 & W & Start Address \\
\hline
0x00F0 & 0,1 & \multicolumn{5}{|l|}{DMA\_DESC\_7a} \\
\cline{3-7}
 & & & RD\_POINTER & 127:64 & W & PC Read Pointer \\
 & & & WRAP\_AROUND & 12 & W & Wrap around \\
 & & & READ\_WRITE & 11 & W & 1: fromHost/ 0: toHost \\
 & & & NUM\_WORDS & 10:0 & W & Number of 32 bit words \\
\hline
\multicolumn{7}{|c|}{DMA\_DESC\_STATUS} \\
\hline
0x0200 & 0,1 & \multicolumn{5}{|l|}{DMA\_DESC\_STATUS\_0} \\
\cline{3-7}
 & & & EVEN\_PC & 66 & R & Even address cycle PC \\
 & & & EVEN\_DMA & 65 & R & Even address cycle DMA \\
 & & & DESC\_DONE & 64 & R & Descriptor Done \\
 & & & CURRENT\_ADDRESS & 63:0 & R & Current Address \\
\hline
\multicolumn{7}{|c|}{\ldots} \\
\hline
0x0270 & 0,1 & \multicolumn{5}{|l|}{DMA\_DESC\_STATUS\_7} \\
\cline{3-7}
 & & & EVEN\_PC & 66 & R & Even address cycle PC \\
 & & & EVEN\_DMA & 65 & R & Even address cycle DMA \\
 & & & DESC\_DONE & 64 & R & Descriptor Done \\
 & & & CURRENT\_ADDRESS & 63:0 & R & Current Address \\
\hline
0x0300 & 0,1 & \multicolumn{2}{|l|}{BAR0\_VALUE} &
31:0 & R & Copy of BAR0 offset reg. \\
\hline
0x0310 & 0,1 & \multicolumn{2}{|l|}{BAR1\_VALUE} &
31:0 & R & Copy of BAR1 offset reg. \\
\hline
0x0320 & 0,1 & \multicolumn{2}{|l|}{BAR2\_VALUE} &
31:0 & R & Copy of BAR2 offset reg. \\
\hline
0x0400 & 0,1 & \multicolumn{2}{|l|}{DMA\_DESC\_ENABLE} &
7:0 & W & Enable descriptors 7:0. One bit per descriptor. Cleared when Descriptor is handled. \\
\hline
0x0410 & 0,1 & \multicolumn{2}{|l|}{DMA\_FIFO\_FLUSH} &
any & T & Flush (reset). Any write clears the DMA Main output FIFO \\
\hline
0x0420 & 0,1 & \multicolumn{2}{|l|}{DMA\_RESET} &
any & T & Reset Wupper Core (DMA Controller FSMs) \\
\hline
0x0430 & 0,1 & \multicolumn{2}{|l|}{SOFT\_RESET} &
any & T & Global Software Reset. Any write resets applications, e.g. the Central Router. \\
\hline
0x0440 & 0,1 & \multicolumn{2}{|l|}{REGISTER\_RESET} &
any & T & Resets the register map to default values. Any write triggers this reset. \\
\hline
0x0450 & 0,1 & \multicolumn{5}{|l|}{FROMHOST\_FULL\_THRESH} \\
\cline{3-7}
 & & & THRESHOLD\_ASSERT & 22:16 & W & Assert value of the ToHost programmable full flag \\
 & & & THRESHOLD\_NEGATE & 6:0 & W & Negate value of the ToHost programmalbe full flag \\
\hline
\caption{FELIX register map BAR0}\label{tab:dma_register_map_bar0} \\
\end{longtabu}

\newpage
BAR1 stores registers associated with the Interrupt vector. The offset for BAR1 is usually 0xFBA00000.

\begin{longtabu} to \textwidth {|X[1.5,l]|X[0.8,l]|X[3,l]|X[4,l]|X[1,l]|X[1,l]|X[5,l]|}
\hline
\textbf{Address} &\textbf{PCIe} &\multicolumn{2}{|l|}{\textbf{Name/Field}} &\textbf{Bits} &{\textbf{Type}} &\textbf{Description} \\
\hline
\endhead

\multicolumn{7}{|c|}{Bar1} \\
\hline
\multicolumn{7}{|c|}{INT\_VEC} \\
\hline
0x0000 & 0,1 & \multicolumn{5}{|l|}{INT\_VEC\_0} \\
\cline{3-7}
 & & & INT\_CTRL & 127:96 & W & Interrupt Control \\
 & & & INT\_DATA & 95:64 & W & Interrupt Data \\
 & & & INT\_ADDRESS & 64:0 & W & Interrupt Address \\
\hline
\multicolumn{7}{|c|}{\ldots} \\
\hline
0x0070 & 0,1 & \multicolumn{5}{|l|}{INT\_VEC\_7} \\
\cline{3-7}
 & & & INT\_CTRL & 127:96 & W & Interrupt Control \\
 & & & INT\_DATA & 95:64 & W & Interrupt Data \\
 & & & INT\_ADDRESS & 64:0 & W & Interrupt Address \\
\hline
0x0100 & 0,1 & \multicolumn{2}{|l|}{INT\_TAB\_ENABLE} &
7:0 & W & Interrupt Table enable\newline Selectively enable Interrupts\newline  \\
\hline
\caption{FELIX register map BAR1}\label{tab:dma_register_map_bar1} \\
\end{longtabu}
\newpage
BAR2 stores registers associated with the user logic and example application. The offset for BAR2 is usually 0xFB900000.
\begin{longtabu} to \textwidth {|X[1.5,l]|X[4.5,l]|X[1.5,l]|X[2,l]|X[5,l]|}
	\hline
	\textbf{Offset} & \textbf{Description} &\textbf{Bits}&\textbf{Direction}& \textbf{Fields}\\
	\hline
	0x0000  & REG\_BOARD\_ID & [63:0] & R/W & Board ID Value\\
	\hline
	0x0010  & REG\_STATUS\_LEDS & [7:0] & R/W & Board GPIO Leds\\
	\hline
	0x0020  & GENERIC\_CONSTANTS& [7:0] & R & NUMBER\_OF\_DESCRIPTORS\\
	        & &[15:8] & R & NUMBER\_OF\_INTERRUPTS \\
	\hline
	0x0040  & CARD\_TYPE& [63:0] & R & Integer number of the card, eg 709 for the VC709 \\
	\hline
	0x0300  & REG\_PLL\_LOCK & [0] & R & PLL Locked status\\
	\hline
	0x0310  & REG\_CORE\_TEMPERATURE & [11:0] & R & XADC temperature monitor for the FPGA CORE\newline for Virtex7\newline temp (C)= ((FPGA\_CORE\_TEMP* 503.975)/4096)-273.15\newline KintexUltrascale\newline temp (C)= ((FPGA\_CORE\_TEMP* 502.9098)/4096)-273.8195\newline  \\
	\hline
	0x0320  & REG\_VCCINT & [11:0] & R & XADC voltage measurement VCCINT = (FPGA\_CORE\_VCCINT *3.0)/4096 \\
	\hline
	0x0330  & REG\_VCCAUX & [11:0] & R & XADC voltage measurement VCCAUX = (FPGA\_CORE\_VCCAUX *3.0)/4096 \\
	\hline
	0x0340  & REG\_VCCBRAM & [11:0] & R & XADC voltage measurement VCCBRAM = (FPGA\_CORE\_VCCBRAM *3.0)/4096 \\
	
	\hline
	0x1060  & REG\_INT\_TEST\_2 & any & W & Fire a test MSIx interrupt \#2 \\
	\hline
	0x1070  & REG\_INT\_TEST\_3 & any & W & Fire a test MSIx interrupt \#3 \\
	\hline
	\multicolumn{5}{|c|}{Example application specific registers} \\
	\hline
	0x2000  & REG\_LFSR\_SEED\_0 & [127:0] & R/W & First part of the seed for the LFSR random number generator \\
	\hline
	0x2010  & REG\_LFSR\_SEED\_1 & [127:0] & R/W & Second part of the seed for the LFSR random number generator \\
	\hline
	0x2020  & REG\_APP\_MUX & [0:0] & R/W & Switches between LFSR output or Multiplier output \\
	\hline
	0x2030  & REG\_LFSR\_LOAD\_SEED & [0:0] & R/W & Loads the LFSR seed to the LFSR module \\
	\hline
	0x2040  & REG\_APP\_ENABLE & [1:0] & R/W & Enables the user application \\
	\hline

	\caption{PCIe Engine register map BAR2}\label{tab:dma_register_map_bar2}
\end{longtabu}
\newpage