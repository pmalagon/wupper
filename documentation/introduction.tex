\section{Introduction}
The PCIe Engine is designed for the ATLAS / FELIX project \cite{atlas}, to provide a simple
Direct Memory Access (DMA) interface for the Xilinx  Virtex-7 PCIe Gen3 hard block. The core is not meant to be flexible among different architectures, but especially designed for the 256 bit wide AXI4-Stream interface
\cite{ug761}  of the Xilinx Virtex-7 FPGA Gen3 Integrated Block for PCI Express (PCIe) \cite{xilinxcore}
\cite{pg023}.

The purpose of the PCIe Engine is to provide an interface to a standard FIFO. This FIFO has the same width as the Xilinx AXI4-Stream interface (256 bits) and runs at 250 MHz. The application side of the FPGA design can simply read or write to the FIFO; the PCIe Engine will handle the transfer into Host PC memory, according to the addresses specified in the DMA descriptors.

Another functionality of the Engine is to manage a set of DMA descriptors, with an $address$, a $read/\overline{write}$ flag, the $transfer size$ (number of 32 bit words) and an $enable$ line. These descriptors are mapped as normal PCIe memory or IO registers. Besides the descriptors and the enable line (one per descriptor), a status register for every descriptor is provided in the register map.

For synthesis and implementation of the cores, it is recommend to use Xilinx Vivado 2014.2. The cores (FIFO, clock wizard and PCIe) are provided in the Xilinx .xci format, as well as the constraints file (.xdc) is in the Vivado 2014.2 Format.

For portability reasons, no Xilinx project files will be supplied with the Engine, but a bundle of TCL scripts has been supplied to create a project and import all necessary files, as well as to do the synthesis and implementation. These scripts will be described later in this document.


\newpage